% $HeadURL: https://sbgn.svn.sourceforge.net/svnroot/sbgn/ProcessDiagram/tags/L1V1.3Full/sources/tag.tex $

\subsection{Glyph: \glyph{Tag}}
\label{sec:tag}

A \glyph{tag} is a named handle, or reference, to another EPN (\sect{EPNs}) or compartment (\sect{compartment}).  \glyph{Tags} are used to identify those elements in \glyph{submaps} (\sect{submap}).

\begin{glyphDescription}

\glyphSboTerm Not applicable.

\glyphContainer A \glyph{tag} is represented by a rectangle fused to an empty arrowhead, as illustrated in \fig{tag}.  The symbol should be linked to one and only one edge (\ie it should reference only one EPN or compartment).

\glyphLabel A \glyph{tag} is identified by a label placed in an unbordered box containing a string of characters.  The characters can be distributed on several lines to improve readability, although this is not mandatory.  The label box must be attached to the center of the container. The label may spill outside of the container.

\glyphAux A \glyph{tag} does not carry any auxiliary items. 

\end{glyphDescription}

\begin{figure}[H]
  \centering
  \includegraphics[scale = 0.3]{images/tag}
  \caption{The \PD glyph for \glyph{tag}.}
  \label{fig:tag}
\end{figure}




% The following is for [X]Emacs users.   Please leave in place.
% Local Variables:
% TeX-master: "../sbgn_PD-level1"
% End:

