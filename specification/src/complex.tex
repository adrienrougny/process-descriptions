% $HeadURL$

%%%%%%%%%%%%%%%%%%%%%%%%%%%%%%%%%%%%%%%%%%%%%%%%%%%%%%%%%%%%%%%%%%%%%%
%%%%                   Complex
%%%%%%%%%%%%%%%%%%%%%%%%%%%%%%%%%%%%%%%%%%%%%%%%%%%%%%%%%%%%%%%%%%%%%%

\subsection{Glyph: \glyph{Complex}}\label{sec:complex}

A \glyph{complex} represents a pool of biochemical entities, each composed of other biochemical entities, whether macromolecules, simple chemicals, multimers, or other complexes. The resulting entity may have its own identity, properties and function in an SBGN map.
The \glyph{complex} can be described by the set of \glyph{subunits} (\sect{subunit}) it contains (see \fig{complexSubunits}). This description is entirely optional and is there to assist the user with a visual shorthand about the composition of the complex.
% 

\begin{glyphDescription}

\glyphSboTerm
SBO:0000253 ! non-covalent complex


\glyphIncoming
Zero or more \glyph{production} arcs (\sect{production}).



\glyphOutgoing
Zero or more \glyph{consumption} arcs (\sect{consumption}), \glyph{modulation arcs} (\sect{modulations}), \glyph{logic arcs} (\sect{logicArc}), or \glyph{equivalence arcs} (\sect{equivalenceArc}).


\glyphContainer

A \glyph{complex} is represented by a rectangular shape with cut-corners (that is, an octogonal shape with sides of two different lengths).
If the \glyph{complex} is described by a set of \glyph{subunits}, then its shape should surround those of its \glyph{subunits}, and the size of the cut-corners should be adjusted so that there is no overlap between its shape and those of its \glyph{subunits}.
The shapes of the \glyph{subunits} must not overlap.

\glyphLabel
A \glyph{complex} is identified by a label that is  a string of characters that may be distributed on several lines to improve readability