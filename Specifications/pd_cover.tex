\documentclass{jib}
\newlength{\platz}
\setlength{\platz}{15pt}
\RequirePackage{listings}

\usepackage{changepage} %test, TODO remove

\lstset{%
  basicstyle=\ttfamily,
  fontadjust,
  flexiblecolumns=true,
  frame=L,
  xleftmargin=15pt,
  framesep=5pt,
  emphstyle=\rmfamily\itshape}

%%%%%%%%%%%%%%%%%%%%%%%%%%%%%%%%%%%%%%%%%%%%%%%%%%%%%%%%%%
% JIB Header/Footer
%%%%%%%%%%%%%%%%%%%%%%%%%%%%%%%%%%%%%%%%%%%%%%%%%%%%%%%%%%
%\jibvolume{XX} % insert volume
%\jibissue{X}   % insert issue
%\jibpages{XXX} % insert article ID
%\jibyear{XXXX} % insert year
%\makeHeaderFooter{} % leave as is
%%%%%%%%%%%%%%%%%%%%%%%%%%%%%%%%%%%%%%%%%%%%%%%%%%%%%%%%%%

\begin{document}

%%%%%%%%%%%%%%%%%%%%%%%%%%%%%%%%%%%%%%%%%%%%%%%%%%%%%%%%%%
%
% Title Page
%
%%%%%%%%%%%%%%%%%%%%%%%%%%%%%%%%%%%%%%%%%%%%%%%%%%%%%%%%%%

\begin{jibtitlepage}

\jibtitle{Systems Biology Graphical Notation:\\Process Description language Level~1 Version~2.0}

%We did not provide author(s) nor author footnote(s), please complete as applicable.
% Please make sure to use unique footnote characters for each author
\jibauthor{Adrien Rougny\iref[,]{adrien1}\iref[,]{adrien2}\jibauthorfootnote{*}{To whom correspondence should be addressed. Email: \email{sbgn-editors@googlegroups.com}},
           Vasundra Touré\iref{vasundra},
           Stuart Moodie\iref{stuart},
           Irina Balaur\iref{irina},
           Tobias Czauderna\iref{tobias},
           Hanna Borlinghaus\iref{hanna},
           Ugur Dogrusoz\iref[,]{ugur1}\iref{ugur2},
           Alexander Mazein\iref[,]{irina}\iref{alexander2},
           % Alexander Mazein\iref[,]{irina}\iref[,]{alexander2}\iref{alexander3},
           Andreas Dräger\iref[,]{andreas1}\iref[,]{andreas2}\iref{andreas3},
           Michael L Blinov\iref{michael},
           Alice Villéger\iref{alice}
           Robin Haw\iref{robin},
           Emek Demir\iref[,]{emek1}\iref{emek2},
           Huaiyu Mi\iref{huaiyu},
           Anatoly Sorokin\iref{anatoly},
           Falk Schreiber\iref[,]{hanna}\iref{tobias},
           Augustin Luna\iref[,]{augustin1}\iref{augustin2}
}

\addjibinstitution{adrien1}{Biotechnology Research Institute for Drug Discovery, AIST, Tokyo 135-0064, Japan}
\addjibinstitution{adrien2}{Computational Bio Big-Data Open Innovation Laboratory (CBBD-OIL), AIST, Tokyo 169-8555, Japan}
\addjibinstitution{vasundra}{Department of Biology, Norwegian University of Science and Technology (NTNU), Trondheim, Norway}
\addjibinstitution{stuart}{Eight Pillars Ltd, 19 Redford Walk, Edinburgh EH13 0AG, UK}
\addjibinstitution{irina}{European Institute for Systems Biology and Medicine, CIRI UMR5308, CNRS-ENS-UCBL-INSERM, Université de Lyon, 50 Avenue Tony Garnier, 69007 Lyon, France}
\addjibinstitution{tobias}{Faculty of Information Technology, Monash University, Melbourne, Australia}
\addjibinstitution{hanna}{Department of Computer and Information Science, University of Konstanz, Konstanz, Germany}
\addjibinstitution{ugur1}{Computer Engineering Department, Bilkent University, Ankara 06800, Turkey}
\addjibinstitution{ugur2}{i-Vis Research Lab, Bilkent University, Ankara 06800, Turkey}
\addjibinstitution{alexander2}{Luxembourg Centre for Systems Biomedicine, University of Luxembourg, 6 Avenue du Swing, L-4367 Belvaux, Luxembourg}
\addjibinstitution{andreas1}{Computational Systems Biology of Infection and Antimicrobial-Resistant Pathogens, Center for Bioinformatics Tübingen (ZBIT), 72076 Tübingen, Germany}
\addjibinstitution{andreas2}{Department of Computer Science, University of Tübingen, 72076 Tübingen, Germany}
\addjibinstitution{andreas3}{German Center for Infection Research (DZIF), partner site Tübingen}
\addjibinstitution{michael}{Center for Cell Analysis and Modeling, UConn Health, USA}
\addjibinstitution{alice}{Freelance IT Consultant, UK}
\addjibinstitution{robin}{Ontario Institute for Cancer Research, MaRS Centre, Toronto, Ontario, Canada}
\addjibinstitution{emek1}{Computational Biology Program, Oregon Health \& Science University, Portland, Oregon, USA}
\addjibinstitution{emek2}{Department of Molecular and Medical Genetics, Oregon Health \& Science University, Portland, Oregon, USA}
\addjibinstitution{huaiyu}{Division of Bioinformatics, Department of Preventive Medicine, University of Southern California, Los Angeles, CA 90033, USA}
\addjibinstitution{anatoly}{Institute of Cell Biophysics, Russian Academy of Sciences, 3 Institutskaya Street, Pushchino, Moscow Region, 142290, Russia}
\addjibinstitution{augustin1}{cBio Center, Department of Data Sciences, Dana-Farber Cancer Institute, Boston, MA 02215, USA}
\addjibinstitution{augustin2}{Department of Cell Biology, Harvard Medical School, Boston, MA 02115, USA}

% \iref[,]{imbio}\jibauthorfootnote{*}{To whom
%            correspondence should be addressed. Email:
%            \email{jib.editorial@degruyter.de}}}
\end{jibtitlepage}


% \begin{abstract}
% 
% This document is a short description of the \LaTeX\ \emph{document class} \emph{jib} and its use. It shall be used to submit \LaTeX\ articles to the \emph{Journal of Integrative Bioinformatics} and at the same time may be used to check whether your \LaTeX\ installation can compile output files according to the guidelines of the journal.
% 
% \end{abstract}

% adjusts the width of the abstract, please do not change!
\begin{adjustwidth}{}{1cm}

\abstract{
The Systems Biology Graphical Notation (SBGN) is an international community effort that aims to standardise the visualisation of pathways and networks for readers with diverse scientific backgrounds as well as to support an efficient and accurate exchange of biological knowledge between disparate research communities, industry, and other players in systems biology. SBGN comprises the three languages Entity Relationship, Activity Flow, and Process Description (PD) to cover biological and biochemical systems at distinct levels of detail. PD is closest to metabolic and regulatory pathways found in biological literature and textbooks. Its well-defined semantics offer a superior precision in expressing biological knowledge.

PD represents mechanistic and temporal dependencies of biological interactions and transformations as a graph. Its different types of nodes include entity pools (e.g., metabolites, proteins, genes and complexes) and processes (e.g., reactions, associations and influences). The edges describe relationships between the nodes (e.g., consumption, production, stimulation and inhibition).

This document details Level 1 Version 2.0 of the PD specification, including several improvements, in particular: 1) the addition of the equivalence operator, subunit, and annotation glyphs, 2) modification to the usage of submaps, and 3) updates to clarify the use of various glyphs (i.e., multimer, empty set, and state variable).
}

\end{adjustwidth} % please sdo not change

\textbf{Keywords:} biological network, circuit diagram, SBGN, standard, systems biology, visualisation.

%%%%%%%%%%%%%%%%%%%%%%%%%%%%%%%%%%%%%%%%%%%%%%%%%%%%%%%%%%
%
% Manuscript Content
%
%%%%%%%%%%%%%%%%%%%%%%%%%%%%%%%%%%%%%%%%%%%%%%%%%%%%%%%%%%


%%%%%%%%%%%%%%%%%%%%%%%%%%%%%%%%%%%%%%%%%%%%%%%%%%%%%%%%%%
%
% Introduction 
%
%%%%%%%%%%%%%%%%%%%%%%%%%%%%%%%%%%%%%%%%%%%%%%%%%%%%%%%%%%

% \section{Introduction}
%
% This text is based on the original Instruction for Authors. The most-recent
% version of the Instructions for Authors are available from:
% \url{https://www.degruyter.com/view/j/jib}
%
% \subsection{Aims and Scope} 
% The Journal of Integrative Bioinformatics (JIB) is an
% international open access journal publishing original peer-reviewed research
% articles in all aspects of integrative bioinformatics.
% This includes: molecular databases, information systems and data warehouses,
% integration of data (methods and tools), metabolic and regulatory network
% modeling and simulation, signal pathways and cell control, network analysis,
% medical informatics, biomedicine and biotechnology, integrative approaches for
% drug design as well as integrative data and text mining approaches.
% The journal will only accept papers which present new tools or new aspects of
% already JIB published tools. Furthermore, these tools must be available and
% usable via the internet and free of charge for academicians. For accepted review
% publications you are invited to join JIBtools as an editor for a new topic:
% \url{http://imbio.de/journal/JIBtools}
%
%
% \subsection{Editorial Policy}
%
% Manuscripts are independently reviewed by peers selected by the Editors.
% Decisions are reached as quickly as possible. JIB aspires to notify authors
% within 6 weeks from submission date. When manuscripts are accepted subject to
% revision, the revised manuscript should be returned within 6 weeks. Accepted
% papers are promptly published online as soon as they have been finally
% processed.
%
%
% \subsubsection{Authorship}
%
% Authorship is restricted to those who have made a significant contribution to
% the conceptual design of the study and/or the execution of the study.
%
%
% \subsubsection{Unpublished Material} 
%
% Submission of a manuscript to JIB implies that the work described is not
% copyrighted, published or submitted elsewhere, except in abstract form. The
% corresponding author should ensure that all authors approve the manuscript
% before its submission to JIB.
%
%
% \subsubsection{Ethical conduct of research} 
%
% The authors must describe and confirm safeguards to meet ethical standards. The
% ethics statements for the Journal of Integrative Bioinformatics are based on the
% Committee on Publication Ethics (COPE) Best Practice Guidelines for Journal
% Editors (see \url{http://publicationethics.org}) and the ``Uniform Requirements
% for Manuscripts (URM) Submitted to Biomedical Journals'' of the International
% Committee of Medical Journal Editors (ICMJE – \url{http://www.icmje.org}).
% Where applicable, all authors must confirm in writing that they have complied
% with the World Medical Association Declaration of Helsinki (see
% \url{http://www.wma.net/en/30publications/10policies/b3/}) regarding ethical
% conduct of research involving human subjects. In the Materials and Methods
% Section, or in a separate section, the manuscript should contain a statement
% that the study has been approved by the Ethical Committee of the institution
% where the study was performed, and that the study subjects, or their legal
% guardians, gave informed consent for participation in the study. If preclinical
% studies performed with animals are described, authors must confirm in writing
% that institutional and national standards for the care and use of laboratory
% animals were followed (please consult the International Association of
% Veterinary Editors' Consensus Author Guidelines on Animal Ethics and Welfare for
% further guidance).
% Please also note that JIB uses the plagiarism detection software “iThenticate”
% to check for potential overlaps with prior publications. Any previously
% published material must be referenced appropriately in the manuscript.
%
%
% \subsubsection{Conflict of Interest} 
%
% When authors submit a manuscript, they are responsible for recognizing and
% disclosing financial and/or other conflicts of interest that might bias their
% work and/or could inappropriately influence his/her judgment. Upon submission,
% the author(s) will be required to complete a form to declare any conflicts of
% interest, funding, employment, or leadership and honoraria. This information
% must also be included within the manuscript before the Reference section. Please
% consult our Publication Ethics and Malpractice Statement first; if you do not
% have any conflicts of interest to report, insert the following declaration:
% Conflict of interest statement: Authors state no conflict of interest. All
% authors have read the journal's publication ethics and publication malpractice
% statement available at the journal's website and hereby confirm that they comply
% with all its parts applicable to the present scientific work.
% If no specified acknowledgement is given, the Publishers assume that no conflict
% of interest exists.
%
%
% \subsubsection{Manuscript Submission} 
%
% Manuscripts should be submitted in
% electronic form to our online submission, peer review and production system at
% \url{http://mc.manuscriptcentral.com/dgjib}.
%
%
% \subsection{Preparation of Manuscript}
%
% \subsubsection{Language}
%
% Manuscripts should be written in clear and concise English. Please have your
% text proofread by an English native speaker before you submit it for
% consideration. At the proof stage, only minor changes other than corrections of
% printers' errors are allowed.
%
% \subsubsection{Cover Letter}
%
% Each manuscript should be accompanied by a cover letter containing a brief
% statement by the authors describing the novelty and importance of their
% research.
%
% \subsubsection{Nomenclature}
%
% Authors are asked to follow the recommendations of the
% Syst\`eme International d'Unit\'es (SI):
% \url{http://www.bipm.org/en/publications/si-brochure/}
%
% \subsubsection{General format}
%
% Manuscripts (including tables and figures, table and figure legends, and
% references) should be typed double- spaced with font size 12 letters. Pages
% should be numbered and have margins of 2.5 cm (1 inch) on all sides. Please
% avoid footnotes in the text, use parentheses instead.
%
% \subsubsection{Article types}
%
% JIB welcomes research articles, review papers and workshop contributions within
% the scope of the journal. If you submit a workshop contribution you will be
% asked to further classify your submission.
%
% \subsubsection{General structure of the text body}
%
% Original research articles should be organized into: Title page, Abstract,
% Keywords, List of non-standard abbreviations (if applicable), Introduction,
% Related works, Architecture/Implementation/Workflow, Application, Discussion,
% Acknowledgments (if applicable), Conflict of Interest statement, References,
% Tables and Figures legends.
%
% Review articles should include: Title page, Abstract, Keywords, Body with
% subsections, Acknowledgments (if applicable), Conflict of Interest statement,
% and References.
%
% \subsubsection{Abstract and Keywords}
%
% The first page of the manuscript should contain the Abstract and the Keywords.
% The Abstract should be a single paragraph of not more than 200 words (original
% article or review) which must be comprehensible to readers before they have read
% the paper. Below the Abstract, up to five keywords, which are not part of the
% title, should be given in alphabetical order and separated by semicolons.
%
% \subsubsection{Abbreviations, Variables, Quotations and more \ldots}
%
% Standard abbreviations in the field of Integrative Bioinformatics do not
% necessarily have to be introduced. If you introduce new abbreviations, please
% write them at the first occurrence in italic, like Integrative Bioinformatics
% (IB), and then in the following text just use the abbreviation in standard font.
% Also, special or Latin terms in the text should be written in italic i.
% For quotations, please indicate them like this: ``The 2nd prize was awarded to''
% (1). Please also note the writing of the ordinal number: ``2nd''.
%
% \subsubsection{References}
%
% Reference lists have to be written in Vancouver style, please see:
% \\
% \url{http://guides.lib.monash.edu/citing-referencing/vancouver} 
% \\
% The names of journals should be abbreviated according to the World List of
% Scientific Periodicals.
% \\
% Number the references consecutively in the order in which they appear in the
% text, including tables and figures. In the body text, identify references by
% Arabic numerals in round brackets (1). In the references all authors must be
% included; et al. is not accepted. Do not use italic font in the reference
% section.
%
% \subsubsection{Tables}
%
% Submit tables on separate pages and number them consecutively using Arabic
% numerals. Provide a short descriptive title, column headings, and (if necessary)
% footnotes to make each table self-explanatory. Refer to tables in the text as
% Table 1, etc. Use Table 1, etc. in the table legends. Please indicate in the
% manuscript the approximate position of each table.
%
% \subsubsection{Figures}
%
% Illustrations will be reduced in size to fit, whenever possible, the width of a
% single column. Lettering in all figures within the article should be uniform in
% style, preferably a sans serif typeface, and of sufficient size, so that it is
% readable at the final size of approximately 2 mm. Uppercase letters A, B, C,
% etc. should be used to identify parts of multi-part figures. Cite all figures in
% the text in a numerical order. Indicate the approximate position of each figure.
% Refer to figures in the text as Figure 1, etc. Use Figure 1, etc. in the figure
% legends.
% \\
% Please note that if any of the figures you used are copyrighted, you need to
% obtain permission from the copyright owners to reproduce these figures in JIB.
% You also need to document the copyright permission in the respective figure
% legend. The same holds true for copyrighted tables you use.
%
% \subsubsection{Colour figures}
%
% Authors are encouraged to submit illustrations in color if necessary for their
% scientific content. Publication of color figures is provided free of charge both
% in online and print editions. Line drawings and photographs must be of high
% quality. Please note that faint shading may be lost upon reproduction.
%
% \subsubsection{Figure legends}
%
% Provide a short descriptive title, and (if necessary) footnotes to make each
% figure self-explanatory on separate pages. Explain all symbols used in the
% figures. Remember to use the same abbreviations as in the body-text.
%
% \subsubsection{Note for authors of NIH-funded research}
%
% Walter de Gruyter Publishers acknowledge that the author of an NIH-funded
% article retains the right to provide a copy of the final manuscript to NIH upon
% acceptance for publication or thereafter, for public archiving in PubMed Central
% 12 months after publication in JIB note that only the accepted author's version
% of the manuscript, not the PDF file of the published article, may be used for
% NIH archiving.
%
% \subsection{Contact}
%
% Please contact us for any further questions:
%
% \subsubsection{Scientific contact:}
%
% \email{jib.editorial@degruyter.com}
% (JIB Editorial Office)
%
% \subsubsection{Publisher contact:}
%
% Alexandra Hinz, Journal Manager DE GRUYTER
% \\
% Genthiner Straße 13
% \\
% 10785 Berlin, Germany
% \\
% Tel. +49-30-26005-358 / Fax: +49-30-26005-325
% \\ 
% E-mail:
% \email{alexandra.hinz@degruyter.com}
%
%
% %%%%%%%%%%%%%%%%%%%%%%%%%%%%%%%%%%%%%%%%%%%%%%%%%%%%%%%%%%
% %
% % Related works
% %
% %%%%%%%%%%%%%%%%%%%%%%%%%%%%%%%%%%%%%%%%%%%%%%%%%%%%%%%%%%
% \section{Related works}
%
% This is an example text for a section.
%
% %%%%%%%%%%%%%%%%%%%%%%%%%%%%%%%%%%%%%%%%%%%%%%%%%%%%%%%%%%
% %
% % Architecture/Implementation
% %
% %%%%%%%%%%%%%%%%%%%%%%%%%%%%%%%%%%%%%%%%%%%%%%%%%%%%%%%%%%
% \section{Architecture/Implementation/Workflow}
%
% This is an example text for a section.
%
% %%%%%%%%%%%%%%%%%%%%%%%%%%%%%%%%%%%%%%%%%%%%%%%%%%%%%%%%%%
% %
% % Application
% %
% %%%%%%%%%%%%%%%%%%%%%%%%%%%%%%%%%%%%%%%%%%%%%%%%%%%%%%%%%%
% \section{Application}
%
% This is an example text for a section.
%
% %%%%%%%%%%%%%%%%%%%%%%%%%%%%%%%%%%%%%%%%%%%%%%%%%%%%%%%%%%
% %
% % Discussion
% %
% %%%%%%%%%%%%%%%%%%%%%%%%%%%%%%%%%%%%%%%%%%%%%%%%%%%%%%%%%%
% \section{Discussion}
%
% This is an example text for a section.
%
%
% %%%%%%%%%%%%%%%%%%%%%%%%%%%%%%%%%%%%%%%%%%%%%%%%%%%%%%%%%%
% %
% % Acknowledgements
% %
% %%%%%%%%%%%%%%%%%%%%%%%%%%%%%%%%%%%%%%%%%%%%%%%%%%%%%%%%%%
% \section*{Acknowledgements}
% %
% Place your acknowledgements here.
% %
%
%
% %%%%%%%%%%%%%%%%%%%%%%%%%%%%%%%%%%%%%%%%%%%%%%%%%%%%%%%%%%
% %
% % Conflict of Interest
% %
% %%%%%%%%%%%%%%%%%%%%%%%%%%%%%%%%%%%%%%%%%%%%%%%%%%%%%%%%%%
% \section*{Conflict of interest statement}
% %
% Here you can state any conflicts of interest.
% \\
% Please consult our Publication Ethics and Malpractice Statement. If you do not
% have any conflicts of interest to report, please use this text:
% \\
% ``Conflict of interest statement: Authors state no conflict of interest. All
% authors have read the journal's Publication ethics and publication malpractice
% statement available at the journal's website and hereby confirm that they comply
% with all its parts applicable to the present scientific work.''
%
%
% %%%%%%%%%%%%%%%%%%%%%%%%%%%%%%%%%%%%%%%%%%%%%%%%%%%%%%%%%%
% %
% % LATEX Template Introduction 
% %
% %%%%%%%%%%%%%%%%%%%%%%%%%%%%%%%%%%%%%%%%%%%%%%%%%%%%%%%%%%
% \section{\LaTeX\ Template Introduction}
%
% The regular doecument ends with the Conflict of interest statement, followed by
% the references (see bottom of this document). In this extra section, the \LaTeX\
%  styles are discussed.
%
% Please read these instructions carefully and follow them strictly so that the
% publication process is efficient. The Journal of Integrative Bioinformatics
% (JIB) reserves the right to return submissions that are not prepared in
% accordance with the following instructions and layout guidelines.
%
% In order to conform to the guidelines of JIB, the document class \emph{jib} has
% to be used in the preamble of your \TeX-file.
%
% \begin{lstlisting}
% \documentclass{jib}
% \end{lstlisting}
%
% The document class \emph{jib} has been derived from the standard document class
% \emph{article}. Thus, a runnable version of \LaTeX2e\ should be installed on
% your system. The developed document class requires some \LaTeX\ packages that
% are not included in a standard \LaTeX\ installation. Thus, it may then be
% required to install the potentially missing packages.
%
% In general, the contribution must be clearly structured. This is especially
% important for long manuscripts. Any reasonable structure will be acceptable.
% However, the following arrangement is recommended: Summary, Introduction,
% Methods, Results, Discussion, References.
%
% Please submit the complete \LaTeX\ source files (*.tex, *.bib, images) together
% with a PDF version. PDF files for submission to the Journal of Integrative
% Bioinformatics should be created using the following steps:
%
% \begin{lstlisting}[emph={texfile,dvi,file}] first step:  latex texfile second
% step: dvipdfm dvi-file
% \end{lstlisting}
%
% \subsection{Command Descriptions}
%
% This section describes the various commands, their syntax and meaning.
%
% \subsubsection{Title Page}
%
% The title page has been defined as an environment. Thus, the special commands
% that define the title page can only be used in this environment.
%
% \begin{lstlisting}
% \begin{jibtitlepage}
% ...
% \end{jibtitlepage}
% \end{lstlisting}
%
% The title page contains the following three structures. The first
% is the \emph{title} of the whole document. Second are the
% \emph{authors} who contributed to the article. Third are
% the \emph{institutions} the authors are affiliated with.
%
% \subsubsection{Title Declaration}
%
% \begin{lstlisting}[emph={title}]
% \jibtitle{title}
% \end{lstlisting}
%
% The argument \emph{title} describes the title of the document.
%
% \vspace{\platz}
% \subsubsection{Author Declaration}
%
% \begin{lstlisting}[emph={list, of, authors}]
% \jibauthor{list of authors}
% \end{lstlisting}
%
% The argument \emph{list of authors} is the list of the manuscript's authors.
% Usually it is a comma-separated list of the authors' names. Sometimes it is
% necessary to place markings, that refer from an author to an institution.
% Therefore, the \emph{iref}-statement should be used.
%
% \begin{lstlisting}[emph={separator, label}]
% \iref[separator]{label}
% \end{lstlisting}
%
% Only one reference can be specified with this statement. If more than one
% reference should be added for an author, also more than one
% \emph{iref}-statement has to be set. To define a reference, the label of that
% institution has to be used. For constituting a label for an institution see
% section \ref{sssec_inst_decl}. With the optional argument \emph{separator} a
% separator symbol can be put to the actual reference, in case more than one
% reference has been defined. Therefore, a comma should be used. The
% \emph{default} value of this argument is \emph{no symbol}.
%
% \begin{lstlisting}[emph={separator, marking, footnotetext}]
% \jibauthorfootnote[separator]{marking}{footnotetext}
% \end{lstlisting}
%
% Another statement is the \emph{jibauthorfootnote}-statement. It allows to put a
% footnote to an author with the specified \emph{footnotetext} and the specified
% \emph{marking} as footnote symbol. The optional argument \emph{separator} acts
% as a counterpart to the \emph{separator} argument of the \emph{iref}-statement.
%
% \vspace{15pt}
% \subsubsection{Institution Declaration}
% \label{sssec_inst_decl}
%
% Unlike the other parts of the titlepage declaration, the institutions are
% defined in separate statements. If, for example three institutions should be
% defined, the following command has to be used three times, one time for each
% institution. There are \emph{two variants} to define an institution. These
% variants result from two different cases. In the \emph{first case}, only one
% institution is defined, that refers to all authors. In the \emph{second case},
% more than one institution is defined. Therefore, marks have to be created that
% allow to make references between specific authors and specific institutions.
%
% \begin{lstlisting}[emph={label,institution}]
% \addjibinstitution*{institution}
% \addjibinstitution{label}{institution}
% \end{lstlisting}
%
% The \emph{first statement} with the \emph{asterisk} should be used when there is
% only one institution valid for all authors.
%
% The \emph{second statement} covers the case when there are more than one
% institutions existing. The first argument (\emph{label}) allows to set a label
% for the institution (\emph{institution}), which can then be used to make a
% reference from authors to that institution.
%
% \subsubsection{Document Structure}
%
% To structure the manuscript, please use the \LaTeX2e standard commands of the
% document class \emph{article} which are \emph{section, subsection, paragraph}
% etc.
%
% An abstract is set by using the \emph{abstract environment}.
% \begin{lstlisting}[emph={text}]
% \begin{abstract}
% text
% \end{abstract}
% \end{lstlisting}
%
% \subsubsection{Bibliography}
%
% To make references to books\cite{jibBook}, papers\cite{jibArticle},
% proceedings\cite{jibProceedings} etc. a BibTeX file containing references must
% be provided. In the following example that file is named
% \emph{jib\_references.bib}. The bibliography style \emph{jib} is related to the
% style \emph{unsrt}.
%
% \begin{lstlisting}[emph={marke1,marke2,label1,label2,description1,description2,format}]
% \bibliographystyle{jib}
% \bibliography{jib_references}
% \end{lstlisting}
%
% Please use the \emph{cite}-command in order to place a reference into the text.
% The argument \emph{label} is a label that was defined by an entry in the BibTeX
% file.
%
% \begin{lstlisting}
% \cite{label}
% \end{lstlisting}
%
% \subsubsection{Font}
%
% The document class \emph{jib} uses the standard \LaTeX2e roman font with a
% font-size of 12pt as default font. To emphasise words, the
% \lstinline|\emph{...}| command can be used. Thus, the text encapsulated by
% parentheses is printed in italic font.
%
% \subsection{Inclusion of URLs and E-Mails}
%
% Three commands will be described in that passage. The \emph{email}- and
% \emph{url}-command are a kind of verbatim environment. The \emph{email}-command
% prints the given \emph{e-mail} in a sans serif font. The \emph{url}-command
% prints the given \emph{URL} in a typewriter font. This \emph{URL} can be opened
% by clicking it.
%
% The \emph{href}-command is part of the hyperref-package. The \emph{text}
% represents a hyperlink to the \emph{URL}. Thus, also that \emph{URL} can be
% opened by clicking on the respective \emph{text}.
%
% \begin{lstlisting}[emph={Email,URL,text}]
% \email{Email}
% \url{URL}
% \href{URL}{text}
% \end{lstlisting}
%
% \subsubsection{Tables and Figures}
%
% The document class \emph{jib} uses the \emph{graphicx}-package to include
% figures. Hence, it is possible to include figures in \emph{jpeg} or \emph{png}
% format with the \emph{includegraphics}-command. Other graphics formats like gif,
% tiff and so forth must be converted to \emph{jpeg} or \emph{png} format before
% pictures can be included.
%
% During the inclusion of those formats, it is necessary to define the bounding
% box. Therefore, the tool \emph{ebb} should be used. This software tool creates a
% file, with the name \emph{picture} and the extension \emph{.bb}.
%
% \begin{lstlisting}[emph={picture,ext}] ebb picture.ext
% \end{lstlisting}
%
% Usually, it is sufficient to copy this file into the same directory as the
% corresponding graphics file. But sometimes there are unpredictable circumstances
% and this will not work. In that case it is possible to add another key of the
% \emph{includegraphics}-command (see key \emph{bb} in Fig.~\ref{fig:jpeg_ex}). The
% values of this key can be extracted from the \emph{*.bb}-file, e.\,g.:
%
% \begin{lstlisting}
% %%BoundingBox: 0 0 557 554
% \end{lstlisting}
%
% In the following, the handling of \emph{encapsulated postscript files} should be
% explained. These files can be included the same way as \emph{png}- or
% {jpeg}-files by using the \emph{includegraphics}-command. Another possibility is
% to apply the \emph{epsfig}-command that is part of the \emph{epsfig}-package.
% \emph{Eps}-files themselves already contain information about the bounding box.
% Thus, it do not has to be created manually.
%
% The use of footnotes inside a caption is not recommended.
%
% The following listing shows an example about how the graphics file
% \emph{example\_fig.jpg} is included. As can be seen, it is located in the same
% directory as the \TeX-file. The \emph{caption} and \emph{label}-commands allow
% to set a \emph{caption} and prepare this figure to be referenced by a
% \emph{label}. It is important that the \emph{label}-statement directly follows
% the \emph{caption}-statement, since the \emph{caption}-command sets the counter
% of the figure and not the graphics itself.
%
% \begin{lstlisting}[breaklines=true]
% \begin{figure}
%  \centerline{\includegraphics[scale=0.5,bb=0 0 557 554]{./example_fig.jpg}}
%  \caption{Caption}
%  \label{Label}
% \end{figure}
% \end{lstlisting}
%
%
%
% \begin{figure}[hp]
% \centerline{\includegraphics[scale=0.5,bb=0 0 557
% 554]{./example_fig.jpg}} \caption{This is an example of the
% inclusion of a \emph{jpeg-file}} \label{fig:jpeg_ex}
% \end{figure}
%
% \begin{figure}[hp]
% \centerline{\includegraphics{./example_fig.eps}} \caption{This is
% an example of the inclusion of an \emph{eps-file} with
% \emph{includegraphics}} \label{fig:eps_ex_var1}
% \end{figure}
%
% \begin{figure}[hp]
% \centerline{\epsfig{file=./example_fig.eps,scale=1.0,angle=0}}
% \caption{This is an example of the inclusion of an \emph{eps-file}
% with \emph{epsfig}} \label{fig:eps_ex_var2}
% \end{figure}
% \clearpage
%
% The definition of a table is identically to the definition of a table in the \LaTeX2e standard documentclass \emph{article}.
%
% \begin{table}[hp]
%     \begin{center}
%     \caption{An example for the inclusion of tables}
%     \label{tab:table_ex}
%         \begin{tabular}{|c|c|}
%         \hline
%         Column1 & Columns2\\
%         \hline
%         \hline
%         \ldots & \ldots\\
%         \ldots & \ldots\\
%         \hline
%         \end{tabular}
%     \end{center}
% \end{table}
%
%
% \subsection{Special Characters}
%
% An analysis of articles revealed that often symbols like \textreg, \textcright\
% and \texttrademark are used. Therefore, the document class \emph{jib} provides
% an easy way to insert these symbols. Simply use the commands
% \lstinline|\textreg, \textcright| and \lstinline|\texttrademark|. These commands
% are part of the \emph{textcomp}-package or are derived from commands of the
% \emph{textcomp}-package.
%
%
% \subsection{Reference Style}
%
% The `vancouver.bst' bibliographic style file (for LaTeX/BibTeX) is generated
% with the docstrip utility and modified manually to meet the ``Uniform
% Requirements for Manuscripts Submitted to Biomedical Journals'' as published in
% N Engl J Med 1997;336:309-315. (also known as the Vancouver style).
%
% This specification may be found on the web page of the International Committe of
% Medical Journal Editors:
%
% \url{http://www.icmje.org}
%
% \subsubsection{Copyright Vancouver Style File}
%
% Copyright 2004  Folkert van der Beek
%
% This work may be distributed and/or modified under the conditions of the LaTeX
% Project Public License, either version 1.3 of this license or (at your option)
% any later version.
% The latest version of this license is in
%   \url{http://www.latex-project.org/lppl.txt}
% and version 1.3 or later is part of all distributions of LaTeX version
% 2005/12/01 or later.
%
% This work has the LPPL maintenance status `maintained'.
%
% The Current Maintainer of this work is Folkert van der Beek.
%
% More information and also contact info can be found at this website:
%
% \url{https://www.ctan.org/tex-archive/biblio/bibtex/contrib/vancouver?lang=en}
%
%
% \subsubsection{Usage}
%
% This bibliography style file is intended for texts in ENGLISH
% This is a numerical citation style, and as such is standard LaTeX.
% It requires no extra package to interface to the main text.
% The form of the
%  
% \begin{lstlisting}
% \bibitem
% \end{lstlisting}
%  
%
% entries is
%
% \begin{lstlisting}
% \bibitem{key}...
% \end{lstlisting}
%   
% Usage of 
%
% \begin{lstlisting}
% \cite 
% \end{lstlisting}
%
% is as follows:
%
% \begin{lstlisting}
% \cite{key} ==>>          [#]
% \cite[chap. 2]{key} ==>> [#, chap. 2]
% \end{lstlisting}
%   
% where \# is a number determined by the ordering in the reference list.
% The order in the reference list is that by which the works were originally
%   cited in the text, or that in the database.
%
% To change the reference numbering system from [1] to 1,
% put the following code in the preamble:
%
% \begin{lstlisting}
% \makeatletter % Reference list option change
% \renewcommand\@biblabel[1]{#1} % from [1] to 1
% \makeatother %
% \end{lstlisting}
%
%
%
% %%%%%%%%%%%%%%%%%%%%%%%%%%%%%%%%%%%%%%%%%%%%%%%%%%%%%%%%%%
% %
% % Bibliography
% %
% %%%%%%%%%%%%%%%%%%%%%%%%%%%%%%%%%%%%%%%%%%%%%%%%%%%%%%%%%%
% \addcontentsline{toc}{section}{References}
% \bibliographystyle{vancouver}
% \bibliography{jib_references}
% \nocite{*}

\end{document}
