\chapter{Issues postponed to future levels}\label{sec:postponed}

\section{Multicompartment entities}
\label{sec: unresolved multi-comp ents}

The problem of entities, such as macromolecules, spanning several compartments proved to be a challenge for the community involved in the development of SBGN Process Description Level 1. It was thus decided to leave it for a future Level. It turns out there is at the moment no obvious solution satisfactory for everyone. Three broad classes of solutions have been identified so far:

\begin{itemize}
\item One can systematically locate an \glyph{EPN} in a given \glyph{compartment}, for instance a transmembrane receptor in a membrane. However, the reactions of this entity with entities represented by \glyph{EPN} in other compartments, such as extracellular ligands and second messenger systems, will create artificial transport reactions.
\item One can represent the domains of proteins in different compartments by \glyph{macromolecules}, and link all those macromolecules in a \glyph{complex} spanning several compartments. However, such a representation would be very confusing, implying that the domains are actually different molecules linked through non-covalent bonds.
\item On can accept \glyph{macromolecules} that span several compartments, and represent domains as \glyph{units of information}. Those \glyph{units of information} should then be located in given compartments. To make a full use of such a representation, one should then start and end connecting arcs on given \glyph{units of information}, something prohibited by the current specification.
\end{itemize}

\section{Logical combination of state variable values}

The value of a \glyph{state variable} has to be perfectly defined in \SBGNPDLone. If a state variable can take the alternative values ``A'', ``B'' and ``C'',  one cannot attribute it values such as ``non-A'', ``A or B'', ``any'' or ``none''. As a consequence some biochemical processes cannot be easily represented because of the very large number of states to enumerate. The decision to forbid such a Boolean logic lies in the necessity of maintaining truth path all over an SBGN map. 

\section{Non-chemical entity nodes}

The current specification cannot represent combinations of events and entities. For instance, a variable ``voltage'' cannot be controlled by a difference of concentration between different entities, such as a given ion in both sides of a membrane. 

% \section{\corr{Generics}{}}
%
% \corr{\SBGNPDLone does not provide mechanisms to sub-class \glyph{EPNs}. There is no specific means of specifying that \glyph{macromolecules} or \glyph{nucleic acid features} X1, X2 and X2 are subclasses of X. Therefore, any process that applies to all the subtypes of X has to be triplicated. That situation can easily generate combinatorial explosions of the number of \glyph{EPNs} or \glyph{PNs}.}{}

\section{State and transformation of compartments}

In \SBGNPDLone a \glyph{compartment} is a stateless entity. It cannot carry  \glyph{state variables}, and cannot be subjected to process modifying a state. As a result, a \glyph{compartment} cannot be transformed, moved, split or merged with another. If one \corr{want}{wants} to represent the transformation of a compartment, one has to create the start and end compartments, and represent the transport of all the \glyph{EPNs} from one to the other. This is not satisfactory, and should be addressed in the future.
