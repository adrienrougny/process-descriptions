\section{Syntax}

The syntax of the SBGN \PDl is defined in the form of an incidence matrix. An incidence matrix has arcs as rows and nodes as columns. Each element of the matrix represents the role of an arc in connection to a node. Source (S) means that the arc can begin at that node. Target (T) indicates that the arc can end at that node. Numbers in parenthesis represent the maximum number of arcs of a particular type to have this specific connection role with the node. Empty cells means the arc is not able to connect to the node.

\subsection{Node connectivity}

\begin{center}
\begin{tabular}{||c|c|c|c|c|c|c|c|c|c|c|c|c||}
\hline
\hline
\raisebox{20pt}{$Arc \backslash EPN$} &\vglyph{macromolecule} & \vglyph{simple chemical} & 
\vglyph{unspecified entity} &  \vglyph{multimer} & \vglyph{complex} & 
\vglyph{nucleic acid feature}& \vglyph{tag} & \vglyph{submap terminal} & \vglyph{empty set} & 
\vglyph{perturbing agent} &  \vglyph{submap}\\ \hline 
\glyph{consumption}      & S & S & S & S & S & S &   & & S(1) &  & \\ \hline 
\glyph{production}        & T & T & T & T & T & T &   & & T(1) &  & \\ \hline 
\glyph{modulation}        & S & S & S & S & S & S &   & &  & S & \\ \hline 
\glyph{stimulation}        & S & S & S & S & S & S &   & & & S & \\ \hline 
\glyph{catalysis}          & S & S & S & S & S &   &   & & &   & \\ \hline 
\glyph{inhibition}          & S & S & S & S & S & S &   & & & S & \\ \hline 
\glyph{necessary stimulation} & S & S & S & S & S & S &  & &  & S & \\ \hline 
\glyph{logic arc}          & S & S & S & S & S & S &   &  & &   & \\ \hline 
\glyph{equivalence arc}     & S & S & S & S & S & S & T & T  & & \\ \hline \hline
\end{tabular}
\end{center}

\paragraph*{Additional rules}

\begin{enumerate}
    \item An EPN that is a \glyph{subunit} of a \glyph{complex} can only be linked to a modulation arc.
    \item With the above exception, glyphs that are subunits cannot be connect to any arc glyph.
    \item A \glyph{logic arc} that is linked to an \glyph{equivalence operator} on one side can only be linked to an EPN on the other side.
\end{enumerate}
    
\begin{center}
\begin{tabular}{||c|c|c|c|c|c|c|c|c|c|c||}
\hline
\hline
\raisebox{20pt}{$Arc \backslash PN$} & \vglyph{process}  & \vglyph{omitted process}  & 
\vglyph{uncertain process} & \vglyph{phenotype}  & \vglyph{association}  & \vglyph{dissociation}  & \vglyph{and}  &  
\vglyph{or} & \vglyph{not} & \vglyph{equivalence} \\ \hline 
\glyph{consumption} & T & T &  T & & T    & T(1) &      &      &  &    \\ \hline
\glyph{production}  & S & S & S & & S(1) & S    &      &      &   &   \\ \hline
\glyph{modulation}  & T & T & T & T  &   &      & S(1) & S(1) & S(1) & \\ \hline
\glyph{stimulation} & T & T & T & T &    &      & S(1) & S(1) & S(1) & \\ \hline
\glyph{catalysis}   & T & T & T & T &    &      & S(1) & S(1) & S(1) & \\ \hline
\glyph{inhibition}  & T & T & T &  T &    &      & S(1) & S(1) & S(1) & \\ \hline
\glyph{necessary stimulation}     & T & T & T &  T &    &      & S(1) & S(1) & S(1) & \\ \hline
\glyph{logic arc}   &   &   &   &      & &      & T    & T    & T(1) & T \\ \hline
\glyph{equivalence arc} &   &   &  &    & &      &      &      &    &  \\ \hline \hline
\end{tabular}
\end{center}

\subsection{Containment definition}
\label{sec:containment}

By containment we mean that a glyph can be drawn inside the other glyph. This does not necessarily mean that the glyph ``belongs'' the the containing node, although in some cases it does. In this section the concept of ``belonging'' is referred to as ownership and you can find node ownership in \sect{conceptual-model}. There are two glyphs that allow containment: \glyph{compartment} and \glyph{complex}. The next table describes the relationship between \PD glyphs and these containers. A $+$ means that the element may be drawn within a container. A $-$ means containment is not allowed.

\begin{center}
\tablefirsthead{\hline
\textbf{Glyph $\backslash$ Containers}  & \textbf{\glyph{complex}} & \textbf{\glyph{compartment}}  \\\hline\hline}
\tablehead{\hline
\multicolumn{3}{|l|}{\small\sl continued from previous page}\\\hline
\textbf{Glyph $\backslash$ Containers}  & \textbf{\glyph{complex}} & \textbf{\glyph{compartment}}  \\\hline\hline}
\tabletail{\hline
\multicolumn{3}{|r|}{\small\sl continued on next page}\\
\hline}
\tablelasttail{\hline}

% \topcaption{The Identifying attributes of \PD node glyphs. When a glyph is always unique in a map, this is indicated by the term \emph{instance}. The term \emph{state values} indicates that the values of all the EPN's state variables are used in the definition of its identity.}
\begin{supertabular}{|c|c|c|}
\glyph{unspecified entity}    &         +       &          +         \\ \hline
\glyph{simple chemical}      &         +       &          +         \\ \hline
\glyph{macromolecule}        &         +       &          +        \\ \hline
\glyph{nucleic acid feature}   &         +       &          +      \\ \hline 
\glyph{multimer}             &         +       &          +         \\ \hline
\glyph{empty set}          &         -       &          +          \\ \hline 
\glyph{perturbing agent}          &         -       &          +      \\ \hline
\glyph{phenotype}           &         -       &          +         \\ \hline
\glyph{tag}                  &         -       &          +          \\ \hline
\glyph{submap terminal}    &         -       &   +         \\ \hline
\glyph{state variable}    &         +       &   +               \\ \hline
\glyph{complex}              &         +       &          +       \\ \hline
\glyph{compartment}          &         -       &          +      \\ \hline
\glyph{submap}               &         -       &          +         \\ \hline
\glyph{process}             &         -       &          +         \\ \hline
\glyph{omitted process}      &         -       &          +     \\ \hline
\glyph{uncertain process}    &         -       &          +        \\ \hline
\glyph{association}          &         -       &          +         \\ \hline
\glyph{dissociation}         &         -       &          +         \\ \hline
\glyph{consumption}          &         -       &          +        \\ \hline
\glyph{production}           &         -       &          +          \\ \hline
\glyph{modulation}           &         -       &          +         \\ \hline
\glyph{stimulation}          &         -       &          +          \\ \hline
\glyph{catalysis}            &         -       &          +          \\ \hline
\glyph{inhibition}           &         -       &          +         \\ \hline
\glyph{necessary stimulation}   &         -       &          +     \\ \hline
\glyph{logic arc}            &         -       &          +          \\ \hline
\glyph{equivalence arc}      &         -       &          +      \\ \hline
\glyph{and}                  &         -       &          +         \\ \hline
\glyph{or}                   &         -       &          +          \\ \hline
\glyph{not}                  &         -       &          +         \\ \hline
\glyph{equivalence}                  &         -       &          +         \\ \hline
\hline
\end{supertabular}
\end{center}
